%    File:    overview.tex
%    Author:  Marvin Smith
%    Date:    11/14/2015
%
%    Purpose:  Give an overview of the Appendix A materials.
%



%------------------------%
%-        Header        -%
%------------------------%
\addcontentsline{toc}{chapter}{Appendix A : Common GIS File Format Information}
\chapter*{Appendix A : Common GIS File Format Information}


\section*{Raster Formats}


%--------------------------%
%-         GeoTiff        -%
%--------------------------%
\subsection*{GeoTiff}
GeoTIFF is an extension of the common \emph{Tagged Image File Format} (TIFF) format.  TIFF
is an image container format which contains both the raster pixel data plus image metadata.  
GeoTIFF extends TIFF by providing metadata fields for coordinate systems, geo-referencing, and 
map-projections.  For example, in QGIS or Google Earth, a GeoTIFF can be loaded and rendered 
directly onto a 3D or 2D map.
    
GeoTIFF is one of the most widely used image raster formats and should be considered for many 
imagery projects.  GDAL supports GeoTIFF for essentially all operations.  In addition, the 
ubiquitous nature of the TIFF format means that a GeoTIFF image will be loadable by any TIFF-compliant
image viewer (which is most if not all major tools).



