%    File:    gdal-overview.tex
%    Author:  Marvin Smith
%    Date:    11/9/2015
%


%------------------------------------------------%
%-                     GDAL                     -%
%------------------------------------------------%
\addcontentsline{toc}{section}{GDAL}
\section*{GDAL}

For many applications, this may be the only tool you will ever need.  \index{GDAL}
stands for the \emph{Geospatial Data Abstraction Library}.  GDAL is largely 
a file input/output tool which can manage well over 100 different raster and vector
files.  GDAL also contains \emph{OGR} which can do coordinate conversion
using the \emph{Proj4} library.


\begin{table}[h!]
\begin{tabular}{l c}\hline
Project Website & \url{http://gdal.org} \\\hline
\end{tabular}
\end{table}


%-------------------------------%
%-       Installing GDAL       -%
%-------------------------------%
\addcontentsline{toc}{subsection}{Installing GDAL}
\subsection*{Installing GDAL}


\addcontentsline{toc}{subsubsection}{Windows}
\subsubsection*{Windows}

Window bindings can be downloaded for various MSVC versions here
at \url{http://www.gisinternals.com/}.  This is a great resource
if you need development libraries for a recent or non-standard
VC++ version. 

Another resource for Windows versions of GDAL is
the OSGeo4W distribution.  This is a tool similar
to Cygwin which provides the latest versions of GDAL packaged
with other useful tools such as QGis, GRASS, MapServer and OpenEV. 
Information can be found on their site at \url{https://trac.osgeo.org/osgeo4w/}.


%----------------------%
%-      Mac OS X      -%
%----------------------%
\addcontentsline{toc}{subsubsection}{Mac OSX}
\subsubsection*{Mac OSX}

The recommended method of installing GDAL is through the MacPorts package manager.
There are other packages available, however MacPorts is unique in that it provides both a customizable
installation through the \texttt{variant} flag and it can be updated through
the package manager.

\begin{verbatim}
sudo port install gdal
\end{verbatim}

Current MacPorts version is 2.0.1 and contains many different port variants.  Recommend
using at least the \emph{openjpeg, geos, lzma, netcdf, and poppler} variants.


%--------------------%
%-      Linux       -%
%--------------------%
\addcontentsline{toc}{subsubsection}{Linux}
\subsubsection*{Linux}

GDAL can easily be installed through multiple distributions.  

\noindent\textbf{Ubuntu}

\begin{verbatim}
 apt-get install gdal-bin libgdal-dev 
\end{verbatim}

\noindent\textbf{Fedora/RHEL/Centos}

\begin{verbatim}
yum install gdal-devel gdal-python
\end{verbatim}

\noindent\textbf{Red-Hat Linux 6}

Due to the stable nature of RHEL distributions, it is recommended to 
use the ELGIS repo provided by OSGeo.  Their Yum repositories can be
configured via the following commands. 

\begin{verbatim}
sudo rpm -Uvh http://elgis.argeo.org/repos/6/elgis-release-6-6_0.noarch.rpm
\end{verbatim}

Caveats
\begin{itemize}
\item[] In order to use ELGIS, you must enable the EPEL repository.  Their repos can be configured at \url{http://fedoraproject.org/wiki/EPEL}. 
\item[] Do not enable both ELGIS and PGRPMS.  They are not compatible and if both enabled, will cause conflicts.
\end{itemize}

More information can be found on their site at \url{http://wiki.osgeo.org/wiki/Enterprise_Linux_GIS}.

%--------------------------------%
%-     Building from Source     -%
%--------------------------------%
\addcontentsline{toc}{subsubsection}{Building from Source}
\subsubsection*{Building from Source}

Before installing from source, first read the GDAL documentation 
at \href{http://gdal.org} for your respective system.  GDAL 
has a huge number of dependencies, most of which are optional. 
As a consequence, it may be difficult for new users to know what is needed
for their respective applications.  In general, here are a good set of
packages to consider installing prior to building GDAL.

\begin{table}[h!]
\begin{tabular}{|l|l|l|}\hline
\textbf{Package} & \textbf{Description}   & \textbf{More Info} \\\hline
proj4            & Coordinate conversions & \url{http://trac.osgeo.org/proj/} \\\hline
openjpeg         & JPEG2000 reader        & \url{http://www.openjpeg.org/} \\\hline
\end{tabular}
\end{table}

