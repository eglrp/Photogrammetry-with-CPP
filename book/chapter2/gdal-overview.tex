%    File:    gdal-overview.tex
%    Author:  Marvin Smith
%    Date:    11/9/2015
%


%------------------------------------------------%
%-                     GDAL                     -%
%------------------------------------------------%
\addcontentsline{toc}{section}{GDAL}
\section*{GDAL}

For many applications, this may be the only tool you will ever need.  GDAL
stands for the \emph{Geospatial Data Abstraction Library}.  GDAL is largely 
a file input/output tool which can manage well over 100 different raster and vector
files.  GDAL also contains \emph{OGR} which can do coordinate conversion
using the \emph{Proj4} library.



%-------------------------------%
%-       Installing GDAL       -%
%-------------------------------%
\addcontentsline{toc}{subsection}{Installing GDAL}
\subsection*{Installing GDAL}


\addcontentsline{toc}{subsubsection}{Windows}
\subsubsection*{Window}



%----------------------%
%-      Mac OS X      -%
%----------------------%
\addcontentsline{toc}{subsubsection}{Mac OSX}
\subsubsection*{Mac OSX}

The recommended method of installing GDAL is through the MacPorts package manager.
There are other packages available, however MacPorts is unique in that it provides both a customizable
installation through the \texttt{variant} flag and it can be updated through
the package manager.


%--------------------------------%
%-     Building from Source     -%
%--------------------------------%
\addcontentsline{toc}{subsubsection}{Building from Source}
\subsubsection*{Building from Source}

Before installing from source, first read the GDAL documentation 
at \href{http://gdal.org} for your respective system.  GDAL 
has a huge number of dependencies, most of which are optional. 
As a consequence, it may be difficult for new users to know what is needed
for their respective applications.  In general, here are a good set of
packages to consider installing prior to building GDAL.

\begin{table}[h!]
\begin{tabular}{|l|l|l|}\hline
\textbf{Package} & \textbf{Description}   & \textbf{More Info} \\\hline
proj4            & Coordinate conversions & \url{http://trac.osgeo.org/proj/} \\\hline
openjpeg         & JPEG2000 reader        & \url{http://www.openjpeg.org/} \\\hline
\end{tabular}
\end{table}

