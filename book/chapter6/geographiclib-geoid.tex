%    File:    geographiclib-geoid.tex
%    Author:  Marvin Smith
%    Date:    11/13/2015
%
%    Purpose:  Describe how to use GeographicLib for coordinate conversion and Geoid manipulation.
%


%-------------------------------------------------------------%
%-       Geographic Magnetic Lookup with GeographicLib       -%
%-------------------------------------------------------------%
\addcontentsline{toc}{subsection}{Querying Geo-Magnetic Information with GeographicLib}
\subsection*{Querying Geo-Magnetic Information with GeographicLib }

Prior to GPS, heading information was attained by using a magnetic compass.  Compasses still have a useful purpose and 
as such, knowing about the Earth's magnetic field is equally useful.  GeographicLib has an impressive amount of 
built-in, yet extensible support for magnetic information.  The \texttt{MagneticModel} class in the library provides
this functionality. 

In the demo below, we will be using GeographicLib to take a Geographic Coordinate in decimal degrees
and compute the magnetic information for that position.  Of particular importance to many users will
be the ability to extract the \emph{declination} or angle from North to the Earth's magnetic North Pole.


\inputminted{C++}{../code/chapter6/geographiclib-magnetic-query.cpp}

