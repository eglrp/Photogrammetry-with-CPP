%    File:    spherical-vs-projected.tex
%    Author:  Marvin Smith
%    Date:    11/7/2015
%


%--------------------------------------------%
%-       Spherical Coordinate Systems       -%
%--------------------------------------------%
\addcontentsline{toc}{subsection}{Spherical Coordinates}
\subsection*{Spherical Coordinates}

Most users are familiar with this first class of coordinates, defined as a \emph{spherical}
coordinate system.  This is commonly referred to the \emph{latitude} and \emph{longitude}.
Spherical coordinates in mathematics typically take the form of $r, \theta,$ and $\phi$ where
$r$ is the radius from the center to the coordinate, $\theta$ is the angle on the x,y plan,
and $\phi$ is the angle from the z axis to xy plane.  Figure \ref{fig:figure_1_2} shows
the relationship between Cartesian and Spherical coordinates. 

\begin{figure}[h!]
\includegraphics[width=3in]{chapter1/diagrams/figure_1_2.png}
\caption{Relationship between spherical and cartesian coordinates.}
\label{fig:figure_1_2}
\end{figure}

Now to describe spherical coordinates in a Geographic setting, 
$\theta$ becomes \emph{longitude}, $\phi$ becomes \emph{latitude}, and 
$r$ is modified from the distance from the radius, to the distance from the
\emph{datum}.  Datums will be discussed later.  To continue, consider 
a datum as an ellipsoid which models the Earth for which you can describe
the ``sea-level" or zero elevation.  

Geographic coordinates are commonly described in degrees.  Figure 
\ref{fig:figure_1_1} shows the degrees of latitude and longitude
for the Lake Tahoe region of the United States.

\begin{figure}[h!]
\includegraphics[width=3in]{chapter1/diagrams/figure_1_1.png}
\caption{Shaded relief map of Lake Tahoe showing lines of latitude and longitude.}
\label{fig:figure_1_1}
\end{figure}

%-------------------------------------------%
%-       Projected Coordinate Systems      -%
%-------------------------------------------%
\addcontentsline{toc}{subsection}{Projected Coordinates}
\subsection*{Projected Coordinates}

Expressing coordinates in latitude and longitude is great for navigation 
as the coordinates are easily relatable to the world-wide coordinates. 
The issue however comes when we try to create maps and other products using
spherical coordinates.  Maps require a 2d \emph{Projection} or \emph{mapping}
from the globe to a flat surface.  


%-----------------------------------------------%
%-        Universal Transverse Mercator        -%
%-----------------------------------------------%
\addcontentsline{toc}{subsubsection}{Universal Transverse Mercator}
\subsubsection*{Universal Transverse Mercator (UTM)}




