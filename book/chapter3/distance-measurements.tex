%    File:    distance-measurements.tex
%    Author:  Marvin Smith
%    Date:    11/20/2015
%



%--------------------------------------------------%
%-       Start of Coordinate Conversions          -%
%--------------------------------------------------%
\addcontentsline{toc}{section}{Distance Measurements}
\section*{Distance Measurements}

Computing the distance between two points is yet another unique challenge in 
Geography.  To many, computing distance on a map seems like a simple problem. 
In Cartesian coordinates, the most often used metric is the \emph{Distance Formula}.

Given $P_1 (X,Y)$ and $P_2 (X,Y)$, the Cartesian distance is simply,
\begin{equation}
d = \sqrt{ \left(P_{2_x} - P_{1_x} \right)^2 + \left(P_{2_y} - P_{1_y} \right)^2}
\end{equation}

or more formally, given N-dimensional coordinates $P_1$ and $P_2$,

\begin{equation}
d = \sqrt{ \sum^{N}_{i=1} (P_{2_i} - P_{1_i} )^2}
\end{equation}


The issue with Cartesian coordinates is that they are only appropriate for certain
conditions.  The conditions will be broken down in the following samples.  First however, 
we will show various examples of more standard techniques.


\addcontentsline{toc}{subsection}{Coordinate Distances for Geographic Coordinates}
\subsection*{Coordinate Distances for Geographic Coordinates}


\addcontentsline{toc}{subsubsection}{Great Circle Distances}
\subsubsection*{Great Circle Distances}

The \emph{Great Circle} is the intersection of a plane with a sphere.  This intersection
results in a circle which represents the shortest distance between any two points on
a sphere\cite[p. 108]{Meyer_Book}.  This section is not meant to provide a useful navigational aid, but rather to 
provide the user with a simple computational example.  Great Circle distances are no longer popular in GIS
as Datums are now in Ellipsoids and computational performance now renders the usefulness less relevant. A 
popular equation for this is the \emph{Haversine} formula.  Avoid the technical great circle equation as
it does not perform well under small distances.

The equation is given where Latitude is $phi$, Longitude is $\lambda$, and the Earth's radius is $r$.

\begin{equation}
d_{gc} = 2r \arcsin \left( \sqrt{ \sin^2 \frac{\phi_2 - \phi_1}{2} + \cos{\phi_1} + \cos{\phi_2} \sin^2 \left( \frac{\lambda_2 - \lambda_1}{2} \right) }  \right)
\end{equation}

Here is a C++ example of the Haversine equation. This example was derived partially from \cite[p. 109]{Meyer_Book}.
\inputminted{C++}{../code/chapter3/great-circle-distance.cpp}

\addcontentsline{toc}{subsubsection}{Geodesic Ellipsoid Distances}
\subsubsection*{Geodesic Ellipsoid Distances}


GeographicLib once again is a useful utility for Geodesic distances.  

\inputminted{C++}{../code/chapter3/geographiclib-ellipsoid-distance.cpp}

